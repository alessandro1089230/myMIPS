\documentclass[a4paper,11pt]{article}
\usepackage[utf8]{inputenc}
\usepackage{geometry}
\usepackage{array}
\usepackage{booktabs}
\usepackage{xcolor}
\usepackage{longtable}
\geometry{margin=1in}

\pagecolor[RGB]{156,234,141} % Exact green background color from the image
\title{\textbf{myMIPS Instruction Set Architecture Documentation}}
\author{Led Robster}
\date{\today}

\begin{document}
	
	\maketitle
	
	\section*{Overview}
	myMIPS is an ISA that imitates the classic MIPS architecture. It is a byproduct of my interest in electronics and software and the result of my learning journey. May it be useful for at least one person other than me.
	
	\section*{Instruction Format}
	myMIPS is a 16-bit architetcure. Instructions are 16 bits wide, data is 16 bits wide. 
	Describe the instruction format, including fields such as opcode, operands, and immediate values. Use a table to illustrate this.
	
\begin{table}[h!]
	\centering
	\setlength{\arrayrulewidth}{0.8mm} % Thicker table lines
	\setlength{\tabcolsep}{3pt} % Reduced space between columns
	\renewcommand{\arraystretch}{1.4} % Increased row height
	\begin{tabular}{|c|c|c|c|c|c}
		\cline{1-5}
		\textbf{15--12} & \textbf{11--9} & \textbf{8--6} & \textbf{5--3} & \textbf{2--0} & \\ \cline{1-5}
		opcode & rs & rt & rd/shamt & Fcode & R-format\\ \cline{1-5}
		opcode & rs & rt & \multicolumn{2}{c|}{offset/immediate} & I-format \\ \cline{1-5}
		opcode & \multicolumn{4}{c|}{address} & J-format\\ \cline{1-5}
	\end{tabular}
	\caption{Instruction Format Example}
	\label{tab:instruction_format}
\end{table}
	
	Formal meaning. \\
	In R-format: rs = alu\_op1, rt = alu\_op2, rd = dest reg  \\
		\hspace*{2em} In R-format SLL or SRL : rs = des reg, rt = alu\_op1, shamt = alu\_op2  \\
	In I-format: rt = dest reg, rs = alu\_op1, imm = alu\_op2 ;  \\
		\hspace*{2em} In I-format BEQ : rs = alu op1, rt = alu op2
	
	
	\section*{Instruction Set}
	List all instructions, grouped by category (e.g., arithmetic, logical, memory, control). Include opcode, function codes, and descriptions.
	
	\subsection*{Arithmetic Instructions}
	Instructions involving arithmetic operators +, -.
	
	\begin{longtable}{|l|l|l|l|p{6cm}|l|}
		\hline
		\textbf{Format} & \textbf{Instruction} & \textbf{Opcode} & \textbf{Fcode} & \textbf{Description} & \textbf{Example (CASM)}\\
		\hline
		R & ADD                  & 0000         & 000            & Add rs and rt, store in rd & ADD \$rd, \$rs, \$rt \\
		\hline
		R & SUB                  & 0000         & 001            & Subtract rt from rs, store in rd & SUB \$rd, \$rs, \$rt \\
		\hline
		I & ADDI                  & 0000         & 010            & Add rt and immediate, store in rs & ADDI \$rs, \$rt, d'10\\
		\hline
	\end{longtable}
	
	\subsection*{Logical Instructions}
	Instructions involving logical operators \&, $|$, $<<$, $>>$.
	\begin{longtable}{|l|l|l|l|p{6cm}|l|}
		\hline
		\textbf{Format} & \textbf{Instruction} & \textbf{Opcode} & \textbf{Fcode} & \textbf{Description} & \textbf{Example (CASM)} \\
		\hline
		R & AND                  & 0000         & 010            & Bitwise AND rs and rt, store in rd & AND \$r2, \$r1, \$r3\\
		\hline
		R & OR                   & 0000         & 011            & Bitwise OR rs and rt, store in rd & OR \$r2, \$r1, \$r3\\
		\hline
		R & SLL                  & 0000         & 101            & Shift-left-logical rt and shamt, store in rs & SLL \$rs, \$rt, shamt\\
		\hline
		R & SRL                  & 0000         & 110            & Shift-left-logical rt and shamt, store in rs & SRL \$rs, \$rt, shamt \\
		\hline
	\end{longtable}
	
	\subsection*{Comparison Instructions}
	Instructions involving relational operators $>$, ==.
	\begin{longtable}{|l|l|l|p{6cm}|l|}
		\hline
		\textbf{Instruction} & \textbf{Opcode} & \textbf{Fcode} & \textbf{Description} & \textbf{Example (CASM)} \\
		\hline
		SLT                  & 0000         & 100            & Set rd=1 if rs$<$rt & SLT \$rd, \$rs, \$rt\\
		\hline
		SLTI                   & 0011         & ---            & Set rt=1 if rs$<$IMM & SLTI \$rs, \$rt, IMM\\
		\hline
	\end{longtable}
	
	
	\section*{Control Flow Instructions}
	List instructions used for branching and jumps, including encoding and behavior.
	
	\begin{longtable}{|l|l|l|p{6cm}|l|}
		\hline
		\textbf{Instruction} & \textbf{Opcode} & \textbf{Fcode} & \textbf{Description} & \textbf{Example (CASM)} \\
		\hline
		BEQ                  & 0110         & ---            & Branch to PC+OFFSET+1 if rs == rt  & BEQ $rs, $rt, OFFSET\\
		\hline
		J                  & 0111         & ---            & Jump to ADDR & J h'F\\
		\hline
		JAL                  & 1000         & ---              & Same as J but PC+1 is stored in \$ra  & JAL h'F\\
		\hline
		JR                  & 0000         & ---              & Jump to address in register rs & JR \$rs\\
		\hline
	\end{longtable}
	
	\section*{Memory Instructions}
	Load/store instructions to access data memory.
	
	\begin{longtable}{|l|l|l|p{6cm}|l|}
		\hline
		\textbf{Instruction} & \textbf{Opcode} & \textbf{Fcode} & \textbf{Description} & \textbf{Example (CASM)}\\
		\hline
		LW                   & 0100         & ---            & Load RAM[rs+OFFSET] in rt & LW \$rs, \$rt, OFFSET\\
		\hline
		SW                   & 0101         & ---            & Store rt in RAM[rs+offset] & SW \$rs, \$rt, OFFSET\\
		\hline
	\end{longtable}
	
	\section*{Register Set}
	Describe the registers available in your ISA, including their names, sizes, and special purposes.
	
	\begin{table}[h!]
		\centering
		\begin{tabular}{|c|c|p{8cm}|}
			\hline
			\textbf{Register Name} & \textbf{Size (bits)} & \textbf{Purpose/Description} \\
			\hline
			r0                    & 32                   & Always zero \\
			\hline
			r1-r7                 & 32                   & General Purpose\\
			\hline
			x2                    & 32                   & Stack pointer \\
			\hline
			r15                   & 32                   & Return address \\
			\hline
		\end{tabular}
		\caption{Register Set}
		\label{tab:register_set}
	\end{table}
	
		\section*{Encoding Examples}
	Provide encoding examples for a few representative instructions.
	
	\begin{table}[h!]
		\centering
		\begin{tabular}{|l|l|l|l|l|}
			\hline
			\textbf{Instruction} & \textbf{Opcode} & \textbf{rs1} & \textbf{rs2} & \textbf{Binary Encoding} \\
			\hline
			ADD x1, x2, x3       & 0110011         & 00010         & 00011         & 0000000 00010 00011 000 00001 0110011 \\
			\hline
			SUB x1, x2, x3       & 0110011         & 00010         & 00011         & 0100000 00010 00011 000 00001 0110011 \\
			\hline
		\end{tabular}
		\caption{Encoding Examples}
		\label{tab:encoding_examples}
	\end{table}
	
	\section*{Additional Notes}
	
	\textbf{CASM Immediate} : immediates can be written in the following format b'XXX, h'YYY, d'ZZZ. Depends on the number base preferred. b' is for binary format, h' for exadecimal format and 'd for decimal format. Immediates are 6-bits long, writing anything higher than 2**6-1 will result in truncation. \\
	Examples: b'111; d'10; h'3F
	
\end{document}
